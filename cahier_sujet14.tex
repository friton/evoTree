\documentclass[a4paper]{article}
\usepackage[francais]{babel}
\usepackage[utf8]{inputenc}
\usepackage[toc,page]{appendix} 
\usepackage[T1]{fontenc}
\usepackage{graphicx}
\usepackage{fancyhdr}
\pagestyle{fancy}

% définir les entêtes et pieds de page
\lhead{Master Bioinformatique}
\rhead{Université Claude Bernard Lyon 1}
\renewcommand{\footrulewidth}{0.4pt}
\lfoot{UE Projet 2}
\rfoot{Février 2017}

% définir le titre et les auteurs sur la couverture
\title{{\sc \large Cahier des charges}\\
\bf Un site web interactif pour une meilleure compréhension de l’histoire évolutive des espèces}
\author{ShangNong {\sc Hu}\and Valentin {\sc Reymond}\and Grégoire {\sc Siekaniec}\and Krystian {\sc Valenducq}}
\date\today

% pour que la numérotation commence à partir de la deuxième page
\setcounter{page}{0}


\begin{document}

\begin{figure}[!t]
	\centering
	\includegraphics[width=6cm]{ucbl.png}
	\hspace{\fill}
	\includegraphics[width=2cm]{lbbe.png}
\end{figure}

\maketitle
% couverture sans numérotation
\thispagestyle{empty}

\begin{figure}[!b]
	\centering
	\includegraphics[width=6cm]{logo.png}
\end{figure}

\newpage

\tableofcontents
\newpage


\section{Contexte biologique}
	La vie sur Terre est apparu il y a quelques milliards d'années. Il a fallut de nombreuses étapes pour que cette grande histoire parvienne jusqu'à l'Homme et ses espèces contemporaines. Cette histoire est parsemée de nombreuses péripéties qui ont résulté par l'apparition et la disparition de très nombreuses espèces. Ces variations sont dues à de nombreuses extinctions qui ont eu lieu au cours des temps géologiques. En revanche ces bouleversements dans l'histoire de la Vie restent à l'heure actuelle très floues quant à leur intensité et leur origine.

	Les trois chercheurs Damien M. de Vienne, Jean-Pierre Flandrois et Laurent Guéguen ce sont interessé à la sensibilisation de ces grandes extinctions sur la biodiversité pour le grand public. L'idée serait de retracer ces événements à l'aide d'un site web interactif, qui pourra par la suite être exploité au sein de musées. 

\section{Analyse des besoins}


	\subsection{Besoins spécifiques du projet}

		

	\subsection{Caractéristiques du logiciel à mettre en place}

		

	\subsection{Ordre de priorité}

		

	\subsection{Outils utilisés pour répondre aux besoins}

			
\section{Analyse de l'existant}

	\subsection{TimeTree}

	\subsection{Onezoom}

	\subsection{lifemap}

\section{Déroulement du projet}
	
	\subsection{Planification}

		
		\subsubsection{Construction du coeur du programme}

			

		\subsubsection{Construction d’une interface graphique}

			

		\subsubsection{Construction des fonctionnalités facultatives}

			

	\subsection{Assurance qualité et tests du programme}

		

\section{Contraintes}

	\subsection{Contrainte temporelle}
		La date de rendu du cahier de charge est prévue le vendredi 17 février.
		La soutenance est prévue provisoirement le vendredi 24 mars, dont le délivrable devrait être rendu 2 jours avant.
		Le projet doit être développé dans une période de 1 mois.

	\subsection{Contraintes techniques}

		

	\subsection{Autres contraintes}

\section{Ouverture}
		
\newpage

\begin{appendices} 
	
\end{appendices} 

\end{document}
